\begin{tabular}{ll}
    \textbf{Navn:} & Arne Magnus Tveita Løken\\
    \textbf{Veileder:} & Tor Nordam (Førsteamanuensis II, Institutt for fysikk;
    Forsker, SINTEF Ocean)\\
    \textbf{Biveileder:} & Ingve Simonsen (Professor, Institutt for fysikk)\\
\end{tabular}

\subsubsection*{Introduksjon}

I mars 2011 traff Tohoku-tsunamien den japanske kysten. Foruten det store
antallet dødsfall medførte tsunamien, sammen med jordskjelvet som forårsaket
den, en katastrofal kjernefysisk nedsmelting i Fukushima-reaktoren
\parencite{atomic2015fukushima}. I
november 2017 medførte en rekke vulkanutbrudd i den indonesiske provinsen
Bali omfattende evakuering av lokalbefolkningen, i tillegg til at mange
nærliggende flyplasser ble stengt. De mange kansellerte flyvningene etterlot
tusenvis av passasjerer strandet på bakken \parencite{kapoor2017bali}. I januar
2018 kolliderte den iranskeide oljetankeren Sanchi med et fraktskip fra Hong
Kong i Øst-Kina-havet \parencite{obayashi2018stricken}.
Tankeren havarerte, store mengder olje ble spredt over havoverflaten og enda
større mengder sank til havbunnen sammen med vrakrestene, hvor oljen truer
med å forurense også dypet av havet dersom beholderne gir etter for det
undersjøiske trykket.

En fellesnevner for de tre ovennevnte naturkatastrofene er at materiale
ble sluppet ut i naturen fra det som kan betraktes som punktkilder. Å predikere
hvor de omsluttende havstrømmene eller de atmosfæriske vindsystemene fraktet
forurensningene var --- og er --- uhyre viktig for å kunne begrense potensielle
humanitære så vel som naturtragedier. Den konvensjonelle måten å gjøre dette på,
er å modellere vind- eller havstrømningsmønstre for å forutse banene
forurensningspartikler vil følge, ved hjelp av numeriske metoder. De
resulterende prediksjonene vil dog være svært sensitive til små endringer i
forurensningspartiklenes utslippstidspunkt og -sted. For å forsøke å håndtere
partiklenes høye grad av sensitivitet til deres initialbetingelser kan en
benytte ulike modeller for det underliggende transportsystemet, med økende
romlig og temporal oppløsning --- men, for komplekse transportsystemer medfører
denne typen tilnærming ofte en beregningskostnad som raskt vokser utover de
tilgjengelige midlene, hva gjelder beregningstid eller minne.

I mange tilfeller vil dog mikroskopiske detaljer i de underliggende
transportfenomenene være uvesentlige i forhold til de overordnede tendensene
i systemet. Dette betyr at en mindre ambisiøs tilnærming, med hovedfokus på
å forstå de makroskopiske trendene i transportfenomenet, ofte er rimelig.
Ved årtusenskiftet så konseptet om \emph{Lagrange-koherente strukturer} dagens
lys, med utspring i skjæringspunktet mellom ikkelineær dynamikk --- de
matematiske prinsippene som utgjør grunnlaget for kaosteori --- og
fluiddynamikk \parencite{haller2000lagrangian}. Denne typen strukturer
utgjør et nytt rammeverk for å forstå transportfenomener i en rekke
strømningssystem. Lagrange-koherente strukturer kan beskrives som
dynamiske <<landskap>> i et mangedimensjonalt rom, som dikterer makroskopiske
flytmønstre i dynamiske system. Mer spesifikt definerer denne typen strukturer
grensesnittene i dynamisk distinkte, invariante regioner. Fra fluiddynamikken
kjennetegnes invariante regioner som domener hvor alle partikkelbaner som
begynner innenfor domenet forblir der inne, selv om domenet kan bevege seg
og deformeres over tid. Så, kort fortalt; Lagrange-koherente strukturer
muliggjør prediksjoner for fremtidige tilstander i flytsystemer.

\subsubsection*{Bakgrunn}

Det underliggende teoretiske rammeverket for Lagrange-koherente strukturer
er veletablert. Ved hjelp av variasjonsteori har nødvendige og
tilstrekkelige eksistensbetingelser for slike strukturer blitt identifisert,
i prinsipp gjeldende for et vilkårlig (heltallig) antall dimensjoner
\parencite{haller2010variational}. Til dags dato (og så vidt undertegnede vet)
har dog ikke en god numerisk implementasjonsmetode for flere av disse
betingelsene blitt beskrevet i litteraturen.
%Følgelig har flere forfattere
%nøyd seg med å benytte såkalte Lyapunoveksponenter som indikatorer på
%Lagrange-koherente strukturer, se f.eks. \textcite{olascoaga2008tracing} eller
%\textcite{shadden2005definition}. Denne tilnærmingen er dog langt fra feilfri,
%selv for konseptuelt simple strømningssystem resulterer bruk av
%Lyapunoveksponenter i både falske positive og falske negative
%\parencite{haller2010variational}.

For todimensjonale strømninger forenkles flere av eksistensbetingelsene
vesentlig, og denne typen systems Lagrange-koherente strukturer kan
prinsipielt beskrives som kurver i et plan. I en relativt nylig publikasjon
anvendes de ovennevnte variasjonsprinsippene for å identifisere
Lagrange-koherente strukturer i modellstrømning så vel som eksperimentelle data
\parencite{farazmand2012computing}. Utfordringen med denne artikkelen er at
forfatterne utelater en fullstendig beskrivelse av hvordan de tar hånd om
de mest numeriske utfordrende av eksistensbetingelsene. I en nyere artikkel
betraktes tredimensjonale strømningssystem, hvor de Lagrange-koherente
strukturene utgjør overflater i rommet \parencite{oettinger2016autonomous}.
Ei heller her presenterer forfatterne en fullstendig beskrivelse av deres
implementasjon av flere av strukturenes eksistensbetingelser. Videre presenterer
de noen av utfordringene ved å utvikle tredimensjonale flater numerisk, og
henviser videre til topologiske studier innenfor dette fagfeltet. Hvorvidt
enkelte metoder egner seg bedre hva gjelder å identifisere Lagrange-koherente
strukturer kommer ikke frem i artiklene.

Med utgangspunkt i det ovenstående, virker det å være stort rom for forskning
innen utvikling av gode og effektive numeriske rutiner for å beregne
Lagrange-koherente strukturer i både to og tre romlige dimensjoner, som for
eksempel med utgangspunkt i modelldata for havstrømninger eller atmosfæriske
vindmønstre. Lagrange-koherente strukturer kan i prinsippet anvendes for å
analysere generiske strømningsfenomen, som i tillegg til de tidligere nevnte
naturfenomenene kan omfatte blodtransport i hjertet, smitte av lakselus
mellom oppdrettsanlegg, eller utvikling av menneskelige forsamlingsmønstre ---
fantasien setter grensene. Med andre ord vil mer forskning innen numeriske
rutiner for å beregne Lagrange-koherente strukturer ha stort nedslagsfelt innen
en rekke fagdisipliner.

\subsubsection*{Mål}

Arbeidet med Lagrange-koherente strukturer vil være en forlengelse av
arbeidet jeg allerede har gjort, og planlegger å gjennomføre, hhv i forbindelse
med fordypningsprosjekt og masteroppgave. Arbeidets fokus vil først og fremst
være utvikling av gode numeriske metoder for beregning av Lagrange-koherente
strukturer i generelle strømningssystem, snarere enn en spesifikk anvendelse.

Konkrete delmål vil være å undersøke og utvikle numeriske metoder for de ulike
stegene som kreves for å identifisere Lagrange-koherente strukturer. Dette
inkluderer:
\begin{itemize}[noitemsep]
    \item Beregning av Cauchy-Greens deformasjonstensor
    \item Numerisk integrasjon av baner i diskontinuerlig vektorfelt
    \item Identifikasjon av lokalt maksimalt tiltrekkende/frastøtende linjer og
        flater
\end{itemize}
Ulike måter å beregne Cauchy-Green-tensoren spiller en viktig rolle i de
Lagrange-koherente strukturenes variasjonsteori; blant annet er dens egenverdier
og -vektorer sentrale i flere av strukturenes eksistensbetingelser. Den første
av to sentrale metoder benytter endelige differansemetoder til å estimere
deformasjonstensoren fra en numerisk beregnet strømningsavbildning, og har blitt
benyttet til identifikasjon av Lagrange-koherente strukturer i to
dimensjoner \parencite{farazmand2012computing, loken2017}. En alternativ
metode innebærer å numerisk løse et koblet sett med differensialligninger, hvis
sluttresultat kan analyseres ved en SVD-dekomposisjon for å beregne tensorens
egenverdier og -vektorer. Den siste metoden er i litteraturen foreslått
benyttet i tre dimensjoner, og hevdes å gi bedre presisjon i de resulterende
Lagrange-koherente strukturene, uten at dette later til å ha blitt undersøkt
i detalj \parencite{oettinger2016autonomous}.

Lagrange-koherente strukturer kan vises å være (tangentielle til) baner i
egenvektorfeltet assosiert med en av egenverdiene til Cauchy-Greens
deformasjonstensor. Siden egenvektorer bare er definert opp til en 180-graders
rotasjon må en lokalt korrigere orienteringen til vektorfeltet, før
interpolasjon til vilkårlig posisjon, for hvert steg i integrasjonen av en bane.
Denne komplikasjonen tatt i betraktning er det uavklart hvilken orden av
interpolasjon som gir den beste balansen mellom regnetid og presisjon;
høyere ordens interpolasjon krever at flere punkter blir korrigert og tatt med
i betraktningen.

Utvelgelsen av lokalt maksimalt tiltrekkende eller frastøtende linjer og flater
er det siste steget i identifikasjonen av hyperbolske Lagrange-koherente
strukturer i henholdsvis to og tre dimensjoner. Metoder for slik utvelgelse er
skissert i litteraturen, se hhv. \textcite{farazmand2012computing} og
\parencite{oettinger2016autonomous}, men mangelfullt beskrevet. I to dimensjoner
har den foreslåtte metoden i tillegg vist seg å være svært sensitiv til flere
fritt valgbare parametere \parencite{loken2017}. I tre dimensjoner beskrives en
metode basert på Poincar\'{e}-avbildninger, og metoden er demonstrert for et
analytisk testsystem med periodiske grensebetingelser. For å kunne identifisere
Lagrange-koherente strukturer i reelle systemer, det være seg strømning i
atmosfæren, en fjord eller et hjertekammer, er man avhengig av en metode som
ikke krever periodiske grensebetingelser.

Selv om fokuset i prosjektet vil være metodeutvikling, vil det være ønskelig å
undersøke en eller flere konkrete anvendelser. Et håndfast mål, hvilket også
kan utgjøre hovedinnholdet i en vitenskapelig
publikasjon, vil være å beregne Lagrange-koherente strukturer
for norske farvann, først i to og så tre dimensjoner, basert på modelldata fra
Meteorologisk institutt. Meteorologisk institutt publiserer daglig et
strømvarsel for norskekysten \parencite{albretsen2011norkyst}. Dette utgjør
blant annet en del av beredskapen ved akutte utslipp av olje eller andre
kjemikalier. En robust og effektiv numerisk metode for automatisk
identifisering av transportbarrierer i de varslede strømdataene vil være av stor
interesse.
